\documentclass[slovene]{beamer}

\usepackage{csquotes}
\usepackage{biblatex}
\addbibresource{viri.bib}
\usepackage[slovene]{babel}
\usepackage{graphicx}
\usepackage{mathtools}
\usepackage{amssymb, amsfonts, amsthm}
\usepackage{booktabs}
\usepackage{multirow}
\usepackage{mdframed}
\usepackage[dvipsnames]{xcolor}
\usepackage{fancyhdr}
\usepackage{ccicons}
\usepackage[hidelinks]{hyperref}

\title{Algoritmi za urejanje podatkov}
\author{\\
    Primož Lah\\
    \small{UL PEF, DU Ma-Ra, 2. letnik}\\
    \small{Matematične tehnologije}\\
    \small{Mentor: dr. Jos\'e Antonio Montero Aguilar}
}
\date{Januar 2026}

\begin{document}

\maketitle

\begin{frame}{Kazalo}
    \tableofcontents[pausesections]
\end{frame}

\section{Definicije}

\subsection{Algoritem za urejanje podatkov}
\begin{frame}{Algoritem za urejanje podatkov}
    \begin{definicija}
        \emph{Algoritem za urejanje podatkov} je postopek, s katerim elemente seznama uredimo po določenem vrstnem redu.
    \end{definicija}
    \pause
    \begin{itemize}
        \item Na urejenem naboru podatkov je izvajanje programov bolj učinkovito.
              \pause
        \item Algoritem za urejanje podatkov mora biti:
              \begin{enumerate}
                  \item Urejen monotono
                  \item Permutacija vhodnega nabora elementov
              \end{enumerate}
    \end{itemize}
\end{frame}

\begin{frame}{Delitev algoritmov}
    \begin{enumerate}
        \item<1-> Tip podatkov
              \begin{itemize}
                  \item Numerični
                  \item Leksikografski
              \end{itemize}
        \item<2-> Lokacija hrambe podatkov med izvajanjem
              \begin{itemize}
                  \item Notranji
                  \item Zunanji
              \end{itemize}
        \item<3> Princip urejanja
              \begin{itemize}
                  \item Zamenjave
                  \item Vstavljanje
                  \item Urejanje po delih
                  \item itd.
              \end{itemize}
              \setcounter{currentenumi}{\theenumi}
    \end{enumerate}
\end{frame}
\begin{frame}{Delitev algoritmov}
    \begin{enumerate}
        \setcounter{enumi}{\thecurrentenumi}
        \item<1-> Prostorska zahtevnost
              \begin{itemize}
                  \item Urejanje na mestu
                  \item Kopiranje nabora elementov
              \end{itemize}
        \item<2> Časovna zahtevnost
              \begin{itemize}
                  \item $O$-notacija
              \end{itemize}
    \end{enumerate}
\end{frame}

\subsection{$O$-notacija}
\begin{frame}{$O$-notacija}
    \begin{definicija}
        \emph{Časovna zahtevnost} je funkcija, ki nam pove, kako se čas izvajanja algoritma povečuje z večanjem nabora vhodnih podatkov.
    \end{definicija}
    \pause
    \begin{definicija}
        \emph{$O$-notacija} je matematični zapis, ki opisuje približno velikost funkcije oz. domene.\\
        V računalništvu $O$-notacija opisuje časovno zahtevnost programa
    \end{definicija}
\end{frame}

\begin{frame}{$O$-notacija}
    \begin{itemize}
        \item Pove, kako se čas izvajanja povečuje s številom vhodnih podatkov, ne pa točnega časa izvajanja
        \item Dva algoritma z enako časovno zahtevnostjo nista nujno enako hitra
    \end{itemize}
    \pause
    \begin{exampleblock}{Primer}
        \begin{table}
            \begin{tabular}{cccc}
                Program & Časovna zahtevnost & t[100] & t[200]                     \\
                \toprule
                $A$     & $O(n^{2})$         & $2 s$  & $4 \cdot (n=2)^2 = 16 s$   \\
                $B$     & $O(n^{2})$         & $2 s$  & $4 \cdot (n=10)^2 = 400 s$ \\
            \end{tabular}
        \end{table}
    \end{exampleblock}
\end{frame}

\begin{frame}{$O$-notacija}
    \begin{itemize}
        \item Je v pomoč programerju za hitrejšo oceno učinkovitosti programa
        \item Program ima tako časovno zahtevnost kot najpočasnejši sestavni del
    \end{itemize}
\end{frame}

\section[Pogosti algoritmi]{Pogosti algoritmi za urejanje podatkov}

\subsection{Bubble sort}
\begin{frame}{Bubble sort}
    \begin{itemize}
        \item \textit{slo.} Mehurčno urejanje
        \item Primerja sosednje pare elementov po velikosti
        \item Intuitiven, a neučinkovit
        \item Časovna zahtevnost:
              \begin{itemize}
                  \item Povprečno: $O(n^{2})$
                  \item V najboljšem primeru: $O(n)$
              \end{itemize}
    \end{itemize}
\end{frame}

\begin{frame}{Delovanje}
    \begin{figure}
        \includegraphics<1>[width=0.65\textwidth]{Images/bubble0.png}
        \includegraphics<2>[width=0.65\textwidth]{Images/bubble1.png}
        \includegraphics<3>[width=0.65\textwidth]{Images/bubble2.png}
        \includegraphics<4>[width=0.65\textwidth]{Images/bubble3.png}
        \includegraphics<5>[width=0.65\textwidth]{Images/bubble4.png}
        \includegraphics<6>[width=0.65\textwidth]{Images/bubble5.png}
        \includegraphics<7>[width=0.65\textwidth]{Images/bubble6.png}
        \includegraphics<8>[width=0.65\textwidth]{Images/bubble7.png}
        \includegraphics<9>[width=0.65\textwidth]{Images/bubble8.png}
        \includegraphics<10>[width=0.65\textwidth]{Images/bubble9.png}
        \caption{Delovanje algoritma Bubble sort}
    \end{figure}
\end{frame}

\subsection{Insertion sort}
\begin{frame}{Insertion sort}
    \begin{itemize}
        \item \textit{slo.} Urejanje z navadnim vstavljanjem
        \item Simulira, kako človek ureja podatke
        \item Časovna zahtevnost: $O(n^2)$
    \end{itemize}
\end{frame}

\begin{frame}{Delovanje}
    \begin{figure}
        \includegraphics<1>[width=0.65\textwidth]{Images/insert0.png}
        \includegraphics<2>[width=0.65\textwidth]{Images/insert1.png}
        \includegraphics<3>[width=0.65\textwidth]{Images/insert2.png}
        \includegraphics<4>[width=0.65\textwidth]{Images/insert3.png}
        \includegraphics<5>[width=0.65\textwidth]{Images/insert4.png}
        \includegraphics<6>[width=0.65\textwidth]{Images/insert5.png}
        \includegraphics<7>[width=0.65\textwidth]{Images/insert6.png}
        \includegraphics<8>[width=0.65\textwidth]{Images/insert7.png}
        \includegraphics<9>[width=0.65\textwidth]{Images/insert8.png}
        \includegraphics<10>[width=0.65\textwidth]{Images/insert9.png}
        \includegraphics<11>[width=0.65\textwidth]{Images/insert10.png}
        \includegraphics<12>[width=0.65\textwidth]{Images/insert11.png}
        \includegraphics<13>[width=0.65\textwidth]{Images/insert12.png}
        \includegraphics<14>[width=0.65\textwidth]{Images/insert13.png}
        \caption{Delovanje algoritma Insertion sort}
    \end{figure}
\end{frame}

\subsection{Quick sort}
\begin{frame}{Quick sort}
    \begin{itemize}
        \item \textit{slo.} Hitro urejanje
        \item Podoben delovanju iskanja z bisekcijo
        \item Časovna zahtevnost:
              \begin{itemize}
                  \item V najslabšem primeru: $O(n^{2})$
                        %\pause
                  \item Povprečno: $O(n\log(n))$
              \end{itemize}
    \end{itemize}
\end{frame}

\begin{frame}{Delovanje}
    \begin{figure}
        \includegraphics<1>[width=0.65\textwidth]{Images/quick0.png}
        \includegraphics<2>[width=0.65\textwidth]{Images/quick1.png}
        \includegraphics<3>[width=0.65\textwidth]{Images/quick2.png}
        \includegraphics<4>[width=0.65\textwidth]{Images/quick3.png}
        \includegraphics<5>[width=0.65\textwidth]{Images/quick4.png}
        \includegraphics<6>[width=0.65\textwidth]{Images/quick5.png}
        \includegraphics<7>[width=0.65\textwidth]{Images/quick6.png}
        \includegraphics<8>[width=0.65\textwidth]{Images/quick7.png}
        \includegraphics<9>[width=0.65\textwidth]{Images/quick8.png}
        \includegraphics<10>[width=0.65\textwidth]{Images/quick9.png}
        \includegraphics<11>[width=0.65\textwidth]{Images/quick10.png}
        \includegraphics<12>[width=0.65\textwidth]{Images/quick11.png}
        \includegraphics<13>[width=0.65\textwidth]{Images/quick12.png}
        \caption{Delovanje algoritma Quick sort}
    \end{figure}
\end{frame}

\subsection{Selection sort}
\begin{frame}{Selection sort}
    \begin{itemize}
        \item \textit{slo.} Urejanje z navadnim izbiranjem
        \item Postopno premika najmanjši element na začetek
        \item Časovna zahtevnost:
              \begin{itemize}
                  \item Povprečno: $O(n^{2})$
                        \pause
                  \item V najboljšem primeru: $O(n^{2})$
              \end{itemize}
    \end{itemize}
\end{frame}

\setbeamercovered{transparent}
\begin{frame}{Delovanje}
    \begin{exampleblock}{Delovanje}
        \begin{table}
            \begin{tabular}{ccc}
                \uncover<1,2,5,8,11>{Preveri prvega}         & \uncover<1,3,6,9,12>{Minimum od ostalih}                   & \uncover<1,4,7,10,13>{Zamenjava}                            \\
                \toprule
                \visible<1->{\textcolor<2->{red}{2} 5 4 3 1} & \visible<3->{\textcolor{red}{2} 4 3 5 \textcolor{blue}{1}} & \visible<4->{\textcolor{blue}{1} 5 4 3 \textcolor{red}{2}}  \\
                \visible<5->{1 \textcolor{red}{5} 4 3 2}     & \visible<6->{1 \textcolor{red}{5} 4 3 \textcolor{blue}{2}} & \visible<7->{1 \textcolor{blue}{2} 4 3 \textcolor{red}{5}}  \\
                \visible<8->{1 2 \textcolor{red}{4} 3 5}     & \visible<9->{1 2 \textcolor{red}{4} \textcolor{blue}{3} 5} & \visible<10->{1 2 \textcolor{blue}{3} \textcolor{red}{4} 5} \\
                \visible<11->{1 2 3 \textcolor{red}{4} 5}    & \visible<12->{X}                                           & \visible<13->{X}                                            \\
            \end{tabular}
        \end{table}
    \end{exampleblock}
\end{frame}

\setbeamercovered{invisible}
\subsection{Counting sort}
\begin{frame}{Counting sort}
    \begin{itemize}
        \item \textit{slo.} Urejanje s štetjem
        \item Neprimerjalen algoritem
        \item Porabi več prostora, a manj časa
        \item Časovna zahtevnost:
              \begin{itemize}
                  \item $O(n+k), k \text{ največji element vhodnega seznama}$
              \end{itemize}
    \end{itemize}
\end{frame}

\setbeamercovered{transparent}
\begin{frame}{Delovanje}
    \begin{exampleblock}{Uporaba začasnega seznama}
        \begin{table}
            \begin{tabular}{cc}
                \uncover<1,3,5,7,8,10>{Začetni seznam}          & \uncover<1,2,4,6,7,9,11>{Začasni seznam}     \\
                \toprule
                \visible<1->{2 5 3 0 2 3 0 3}                   & \visible<2->{0 0 0 0 0 0}                    \\
                \visible<3->{\textcolor{red}{2} 5 3 0 2 3 0 3}  & \visible<4->{0 0 \textcolor{blue}{1} 0 0 0}  \\
                \visible<5->{2 \textcolor{red}{5} 3 0 2 3 0 3}  & \visible<6->{0 0 1 0 0 \textcolor{blue}{1}}  \\
                \visible<7->{\ldots}                            & \visible<7->{\ldots}                         \\
                \visible<8->{2 5 3 0 2 3 \textcolor{red}{0} 3}  & \visible<9->{\textcolor{blue}{2} 0 2 2 0 1}  \\
                \visible<10->{2 5 3 0 2 3 0 \textcolor{red}{3}} & \visible<11->{2 0 2 \textcolor{blue}{3} 0 1} \\
            \end{tabular}
        \end{table}
    \end{exampleblock}
\end{frame}

\begin{frame}{Delovanje}
    Začetni seznam: 2 5 3 0 2 3 0 3
    \begin{exampleblock}{Izpis rezultata}
        \begin{table}
            \begin{tabular}{cc}
                \uncover<1,3,5,7,9,11,13>{Začasni seznam}   & \uncover<1,2,4,6,8,10,12,14>{Odgovor}                 \\
                \toprule
                \visible<1->{2 0 2 3 0 1}                   & \visible<2->{\_ \_ \_ \_ \_ \_ \_ \_}                 \\
                \visible<3->{\textcolor{red}{2} 0 2 3 0 1}  & \visible<4->{\textcolor{blue}{0 0} \_ \_ \_ \_ \_ \_} \\
                \visible<5->{2 \textcolor{red}{0} 2 3 0 1}  & \visible<6->{0 0 \_ \_ \_ \_ \_ \_}                   \\
                \visible<7->{2 0 \textcolor{red}{2} 3 0 1}  & \visible<8->{0 0 \textcolor{blue}{2 2} \_ \_ \_ \_}   \\
                \visible<9->{2 0 2 \textcolor{red}{3} 0 1}  & \visible<10->{0 0 2 2 \textcolor{blue}{3 3 3} \_}     \\
                \visible<11->{2 0 2 3 \textcolor{red}{0} 1} & \visible<12->{0 0 2 2 3 3 3 \_}                       \\
                \visible<13->{2 0 2 3 0 \textcolor{red}{1}} & \visible<14->{0 0 2 2 3 3 3 \textcolor{blue}{5}}      \\
            \end{tabular}
        \end{table}
    \end{exampleblock}
\end{frame}

\setbeamercovered{invisible}
\subsection{Bogosort}
\begin{frame}{Bogosort}
    \begin{itemize}
        \item \textit{slo.} Naključno preverjanje
        \item Po principu \textit{poskusi in preveri}
        \item Časovna zahtevnost:
              \begin{itemize}
                  \item $O(n \cdot n!)$
              \end{itemize}
    \end{itemize}
\end{frame}

\begin{frame}{Delovanje}
    \begin{figure}
        \includegraphics<1>[width=0.65\textwidth]{Images/bogo0.png}
        \includegraphics<2>[width=0.65\textwidth]{Images/bogo1.png}
        \includegraphics<3>[width=0.65\textwidth]{Images/bogo2.png}
        \includegraphics<4>[width=0.65\textwidth]{Images/bogo3.png}
        \includegraphics<5>[width=0.65\textwidth]{Images/bogo4.png}
        \includegraphics<6>[width=0.65\textwidth]{Images/bogo5.png}
        \caption{Delovanje algoritma Bogosort}
    \end{figure}
\end{frame}

\section{Primerjava časovnih zahtevnosti}

\begin{frame}{Časovne zahtevnosti}
    \begin{table}[h]
        \centering
        \begin{tabular}{cc}
            \toprule
            Algoritem      & Povprečna časovna zahtevnost \\
            \midrule
            Bubble sort    & \multirow{3}{*}{$O(n^{2})$}  \\
            Insertion sort &                              \\
            Selection sort &                              \\
            \midrule
            Quick sort     & $n \log(n)$                  \\
            \midrule
            Counting sort  & $O(n+k)$                     \\
            \midrule
            Bogosort       & $O(n \cdot n!)$              \\
            \bottomrule
        \end{tabular}
        \caption{Časovne zahtevnosti predstavljenih algoritmov}
    \end{table}
\end{frame}

\begin{frame}{Primerjava grafov}
    \begin{figure}[h]
        \centering
        \includegraphics[height=0.65\textheight]{Images/Grafi_O.png}
        \caption{Grafi funkcij časovne zahtevnosti}
    \end{figure}
\end{frame}

\section*{Viri}
\begin{frame}{Viri}
    \nocite{*}
    \printbibliography
\end{frame}

\end{document}